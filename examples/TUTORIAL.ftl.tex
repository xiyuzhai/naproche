\documentclass[11pt]{article}
\usepackage{amssymb}
\usepackage{exercises}
\usepackage{url}
\usepackage{../lib/tex/naproche}

\author{Peter Koepke\\University of Bonn}

\title{\textbf{\Naproche{} Tutorial}}

\begin{document}

\newcommand{\Prod}[3]{#1_{#2} \cdots #1_{#3}}
\newcommand{\Seq}[2]{\{#1,\dots,#2\}}
\newcommand{\Finset}[3]{\{#1_{#2},\dots,#1_{#3}\}}
\newcommand{\Primes}{\mathbb{P}}

\maketitle

\section{Introduction}
\Naproche{} (Natural Proof Checking) is a mathematical proof assistant
for texts in the input language ForTheL (Formula Theory Language).
ForThel aims to approximate the
language, structure and appearance of common mathematical texts;
FortheL texts are checked by \Naproche{} for syntactic, ontological
and logical correctness. \Naproche{} is a component of
Isabelle PIDE (Proof Interactive Development Environment) which provides
comfortable text editing and interactive checking through a continuously
running \Naproche{} server.

Ideally, a user edits a mathematical text in \LaTeX{}, and
\Naproche{} automatically checks the correctness of those parts which are
in \verb+\begin+\verb+{forthel}+ ... \verb+\end{forthel}+ 
environments and gives
feedback to the user, similar to a continuous spellchecker.
Although that goal is still far away, the current \Naproche{} offers
the possibility to write university-level mathematical texts 
in a natural language and style whose ForTheL segments are logically
verified by \Naproche{}. In this tutorial and in example texts
which are distributed together with \Naproche{} these segments
are typeset on a grey background.

\Naproche{} is a derivative of the ground-breaking System 
for Automated Deduction (SAD). \Naproche{} still supports the 
ASCII format \path{.ftl} of SAD which allows rapid 
experiments without worrying about LaTeX particulars, and we
use that format for a Quick Start and some exercises. 

This tutorial is an introduction to the principles and
use of the \Naproche{} prover. After the Quick Start and 
some general information we explain 
ForTheL commands and statements and the structuring of ForTheL texts
along a proof of the infinitude of primes. Actually the tutorial is 
a version of Euclid's proof, preceded by 
necessary number-theoretic and set-theoretic preliminaries.
After studying this material, a reader should be able to understand the
other example texts and start writing similar formalizations.
Let us, however, warn the reader that writing formally complete
and unambiguous mathematical texts in a readable natural language 
is subtle and requires ample exercise, as one may already guess by
looking at legal texts.

We suggest that the tutorial is read as pdf and simultaneously followed 
interactively in the Isabelle
PIDE (Proof Interactive Development Environment), with the source
\path{TUTORIAL.ftl.tex} opened in one buffer, using other buffers
for experiments and exercises. Moreover one may use a \LaTeX{} compiler
for displaying \path{.ftl.tex} formalizations.

\section{Quick Start: $1 + 2 = 3$}

Let us prove our first theorem in Isabelle/Naproche:

\begin{itemize}
\item Download and install Isabelle2022 from
\path{https://isabelle.in.tum.de/} (this may take several minutes).

\item Start Isabelle jEdit which may again some time and
start editing a new file \path{temp.ftl}; \path{.ftl} files are recognized
as ForTheL files and are continuously checked by Isabelle/Naproche.

\item Copy the following bit of number theory into \path{temp.ftl}:
\begin{verbatim}
[synonym number/numbers]
Signature. A natural number is a mathematical object.
Signature. 0 is a natural number.
Signature. 1 is a natural number.
Signature. Assume that k,l are natural numbers.
k + l is a natural number.
Definition. 2 = 1 + 1.
Definition. 3 = 2 + 1.
Axiom. Assume that k,l,m are natural numbers.
(k+l)+m=k+(l+m).
Theorem. 1 + 2 = 3.
\end{verbatim}

\item The Output window of Isabelle/Naproche displays the checking
process, ending in \verb+verification successful+ and
some statistics.
Further feedback is given by coloured highlighting of the buffer and by
ballons when hovering over sentences with the mouse pointer.

\item The theorem $1+2=3$ has been verified!
\end{itemize}

\subsection{Explanation of the ForTheL Text}
The above text consists of a sequence of ForTheL ``commands'' that steer its
parsing and logical processing. The ForTheL language is designed
in a way that appears natural to human mathematicians although it is a
completely formal language like a programming language.

\begin{itemize}

\item \verb+[synonym number/numbers]+ is a linguistic command which
identifies the identifiers \verb+nummber+ and \verb+nummbers+ in the
subsequent text. This \textit{allows} to use grammatically correct singular
or plural forms. On the other hand \Naproche{} does not enforce grammatical
correctness.

\item \verb+Signature+ commands define and extend the mathematical
``vocabulary'' of the text. The command
\\\verb+Signature. A natural number is a mathematical object.+\\
has the following effects:
\begin{itemize}
\item The noun phrase ``natural number'' is added to
the vocabulary of the parser; by the previous \verb+[synonym ...]+
command one can also write ``natural numbers'' instead.
\item Internally, a unary predicate symbol
\verb+aNaturalNumber+ is generated, to be used by the first-order
logical processing. One can see this symbol by hovering the mouse
over the command: a pop-up balloon will show a translation to a
first-order formula which contains the subformula
\verb+aNaturalNumber(v0)+ with a variable \verb+v0+ inserted.
\item The first-order translation\\
\verb+forall v0 ((HeadTerm :: aNaturalNumber(v0))+\\
\verb+implies aObject(v0))+\\
is added to internal representation of the text. Note that
\Naproche{} has an inbuilt predicate \verb+aObject+ to encompass
all mathematical objects to be used in a text. This predicate is
addressed by the phrases ``object'' or ``mathematical object''.
The first-order
translation expresses that the collection of natural numbers is
a subcollection of the collection of all mathematical objects.
\item The newly introduced predicate is tagged by \verb+HeadTerm+,
to be used in further processing.
\end{itemize}

\item \verb+Signature. 0 is a natural number.+ extends the
language by a new constant symbol \verb+0+ which is intended to be
a natural number. This can be seen in the translation:
\verb+forall v0+\\
\verb+((HeadTerm :: v0 = 0) implies aNaturalNumber(v0))+.

\item \verb/Signature. Assume that k,l are natural numbers./\\
\verb/k + l is a natural number./\\ 
generates an
internal binary function symbol \verb/+/. The translation
shows that the (universally quantified) variables \verb/k/ and
\verb/l/ are natural numbers, and that the resulting term
\verb/k+l/ is a natural number. So the \verb/+/ operation is
of type $\mathbb{N} \times \mathbb{N} \rightarrow \mathbb{N}$, where
$\mathbb{N}$ is the standard notation for the class / set / type of
natural numbers.

\item The command \verb/Definition. 2 = 1 + 1./ also extends the
language. A new constant symbol \verb/2/ is added together with
an obvious definitional equivalence
\verb/forall v0 ((HeadTerm :: v0 = 2) iff v0 = 1+1)/.

\item The only axiom of our small number theory is added by the
command \\
\verb/Axiom. Assume that k,l,m are natural numbers./\\
\verb/(k+l)+m=k+(l+m)./\\
The first-order translation
includes the guards\\
\verb/((aNaturalNumber(k) and aNaturalNumber(l))/\\
\verb/and aNaturalNumber(m))/\\
for the universally quantified variables, as explained above.

\item Finally
\verb/Theorem. 1 + 2 = 3./ requests that the statement $1+2=3$ be
proved. The system sends the previous first-order
translations as ``hypotheses'' and the thesis ``$1+2=3$'' as
``conjecture'' to the external automated theorem prover (ATP) in
the first-order format TPTP:
\end{itemize}
\begin{footnotesize}
\begin{verbatim}
fof(m_,hypothesis,( ! [W0] : (aNaturalNumber(W0) => aObject(W0)))).
fof(m_,hypothesis,aNaturalNumber(sz0)).
fof(m_,hypothesis,aNaturalNumber(sz1)).
fof(m_,hypothesis,( ! [W0] : ( ! [W1] : ((aNaturalNumber(W0) & 
      aNaturalNumber(W1)) => aNaturalNumber(sdtpldt(W0,W1)))))).
fof(m_,hypothesis,((sz2 = sdtpldt(sz1,sz1)) & ( ! [W0] :
      ((W0 = sdtpldt(sz1,sz1)) => (W0 = sz2))))).
fof(m_,hypothesis,((sz3 = sdtpldt(sz2,sz1)) & ( ! [W0] : 
      ((W0 = sdtpldt(sz2,sz1)) => (W0 = sz3))))).
fof(m_,hypothesis,( ! [W0] : ( ! [W1] : ( ! [W2] :
      (((aNaturalNumber(W0) & aNaturalNumber(W1)) & aNaturalNumber(W2))
      => (sdtpldt(sdtpldt(W0,W1),W2) = sdtpldt(W0,sdtpldt(W1,W2)))))))).
fof(m__,conjecture,(sdtpldt(sz1,sz2) = sz3)).
\end{verbatim}
\end{footnotesize}
\begin{itemize}
\item \Naproche{} uses internal reasoning and the eprover ATP to
derive the conjecture
along the equality chain:
$$1 + 2 = 1 + (1 + 1) = (1 + 1) + 1 = 2 + 1 = 3,$$
as one could see from the eprover proof output.
\end{itemize}

\subsection{Exercises}

\begin{exercise}
Extend the above number theory by further constants and prove
equalities like $2 + 3 = 5$. What about $2 + 2 = 5$? What about
$0+0$? Can
you prove that $2 + 2 \neq 5$? (In the ftl-language, $\neq$ is expressed
by the ASCII combination \verb+!=+.)
\end{exercise}

\begin{exercise}
One can define the order relation $\leq$ in our number theory
by\\
\verb/Definition. k \leq l iff there exists a natural number m/\\
\verb/such that k + m = l./\\
Which axioms for partial or linear orders can now be proved about
\verb+\leq+?
\end{exercise}

\begin{exercise} (Pretyping of variables) In our number theory,
the variables \verb+k,l,m+ in the definition of $+$ or in the
associativity axiom are typed by explicit assumptions. Instead one
can ``pretype'' variables until the end of a file by a command of
the form\\
\verb+Let k,l,m denote natural numbers.+\\
Eliminate the type assumptions by pretyping.
What happens, if a variable does not have a previous
type assumption?
\end{exercise}

\begin{exercise} Experiment with your \verb+.ftl+ texts.
What happens if one exchanges \verb+Definition+ by
\verb+Signature+ and vice versa?
\end{exercise}

\section{ForTheL and \LaTeX}

ForTheL is intended to be faithful to the mathematical
language as found in mathematical books and papers.
These are usually typeset in \LaTeX which in particular
allows the rendering of complex symbolism. This motivates the
integration of ForTheL into \LaTeX. Currently, \Naproche{}
accepts two dialects of the Formula Theory Language:
an ASCII-based one indicated by the .ftl file ending, as used above,
and a LaTeX-oriented version with a .ftl.tex file ending.
The latter can be immediately processed by \LaTeX{} for
high-quality mathematical typesetting.

The development of \Naproche{} is directed towards the \LaTeX-format,
also because typesetting contains semantic information
that may be exploited in the proof checking process.
In \LaTeX, e.g., the variable $a$ is distinguished from the word ``a''
by an italic font. One may also introduce custom \LaTeX{} environments
to signal specific processing of parts of texts.
Translating \verb+.ftl+-files to \verb+.ftl.tex+ is usually
straightforward. ForTheL-constructs like \verb+Definition. ...+
or \verb+Theorem. ...+ obviously correspond to definition or
theorem environments in \LaTeX.

In this tutorial, we mainly use the \verb+.ftl.tex+-format.
The ASCII-format will, however, be supported for some time
since it allows for experimentation with \Naproche{} without the
effort of correct \LaTeX{} syntax. Whereas finished \Naproche{}
formalizations should be written in \LaTeX{} format and be accompanied
by pdf-printouts, the ASCII format might be particularly important
in didactical situations which emphasize logical and mathematical content.

\subsection{Natural Numbers in .ftl.tex}

We introduce the language of arithmetic in the \LaTeX{} dialect:
start a new file with the \LaTeX-ending \path{.ftl.tex}, containing:

\begin{verbatim}
\begin{forthel}

[synonym number/numbers]

\begin{signature}  A natural number is a mathematical object.
\end{signature}

Let $m,n$ denote natural numbers.
\begin{signature} $0$ is a natural number.
\end{signature}

Let $x$ is nonzero stand for $x \neq 0$.

\begin{signature} $1$ is a nonzero natural number.
\end{signature}

\begin{signature} $m + n$ is a natural number.
\end{signature}

\begin{signature} $m * n$ is a natural number.
\end{signature}
\end{forthel}
\end{verbatim}

This file which will also be the beginning of our formalization
of Euclid's theorem on the infinitude of primes.

At the same time it is a \LaTeX{} file which can by typeset as a
\LaTeX{} document in the context of an adequate \LaTeX{} preamble

\begin{verbatim}
\documentclass[11pt]{article}
\usepackage{amssymb}
\usepackage{exercises}
\usepackage{url}
\usepackage{../lib/tex/naproche}

\begin{document}
.....
.....
\end{document}
\end{verbatim}

The \LaTeX style file \path{../lib/tex/naproche} defines the
forthel environment so that ForTheL content, i.e., the
logical core of a text, is typeset against a light grey background:

\begin{greybox}

[synonym number/numbers]

\begin{signature}  A natural number is a mathematical object.
\end{signature}

Let $m,n$ denote natural numbers.

\begin{signature} $0$ is a natural number.
\end{signature}

Let $x$ is nonzero stand for $x \neq 0$.

\begin{signature} $1$ is a nonzero natural number.
\end{signature}

\begin{signature} $m + n$ is a natural number.
\end{signature}

\begin{signature} $m * n$ is a natural number.
\end{signature}
\end{greybox}

\begin{exercise}
Convert the initial formalization of arithmetic to the \LaTeX{} format.
\end{exercise}

\begin{exercise}
Experiment with the \LaTeX{} formalization to see whether greek letters,
letters in other fonts like bold, fracture, ..., other symbols,
possibly with decorations
like $\pi^+$ or $\tilde{\otimes}$ can be used as variables or
operation symbols.
\end{exercise}

\subsection{Literate Formalizations}
Since only content in
\begin{quotation}
\verb_\_\verb_begin{forthel} ... _\verb_\_\verb_end{forthel}_
\end{quotation}
\noindent environments is passed to \Naproche{} one can write arbitrary material
outside those environments. This allows to insert arbitrary comments,
including sectioning commands or even pictures, into
\verb+.ftl.tex+ formalizations.

As was mentiioned before, the \LaTeX{} source \verb+TUTORIAL.ftl.tex+ of this document
is a proper ForTheL text which proof-checks successfully in
\Naproche{}. One can experiment with this text by loading it into the Isabelle-PIDE
and editing the
forthel environments. It may be convenient to
deactivate some of these environments by replacing the outer
\verb_\_\verb_begin_ and \verb_\_\verb_end_ by
\verb_begin_ and \verb_end_.

\begin{exercise}
Augment the \LaTeX formalization of arithmetic so that it
prettyprints with a title, abstract, some comments, and a
small bibliography whilst still being a correct ForTheL formalization.
\end{exercise}

\begin{exercise}
Write a formalization of the real numbers as a ring in a language
that has $+,\cdot,-,0,1$. You may use the ASCII format (observe that
in that format, lines starting with a \verb+#+ are interpreted
as comment).
\end{exercise}


\section{General Principles of \Naproche}

\subsection{Natural Language Processing}

\Naproche interprets ForTheL input texts in first-order logic
and also proposes proof methods to be used by the internal reasoner
of \Naproche{} or by the external ATP.
ForTheL leverages a number of natural language
mechanisms to capture formal content in a compact, user-friendly
and natural way. This corresponds to usual natural language
features, where the phrase ``white horse that belongs
to Mary'' with its adjective, noun and relative sentence
corresponds to a first-order statement like
$$horse(x) \wedge white(x) \wedge property-of(x,Mary)$$
with a (hidden) variable $x$, predicates $horse()$,
$white()$, and $property-of( , )$, and a constant $Mary$.
\Naproche{} extracts this formal context whilst reading
the input sentence by sentence. Previous sentences provide the context
of already introduced language components, in which the
new sentence is to be interpreted.

Indeed the ``white horse'' can be formalized in ForTheL. We use
the ASCII format, since we are mainly developing a language.

\begin{verbatim}
Signature. A human being is a notion.
Signature. Mary is a human being.
Signature. An animal is a notion.
Signature. A horse is an animal.

Let A denote an animal. Let H denote a human being.
Signature. A is white is an atom.
Signature. A belongs to H is an atom.
\end{verbatim}
From these premisses we can show the following triviality:
\begin{verbatim}
Theorem. Every white horse that belongs to Mary
is white.
\end{verbatim}

\begin{exercise}
Inspect this natural language formalization: what is the
meaning of ``notion'' and ``atom''. Why can't you replace
``Every'' by ``Some'' in the Theorem. What is the meaning
of the error message for the theorem ``Mary is white''?
Change the example so that horses and Mary can all be
potentially white.
\end{exercise}

\begin{exercise}
Formalize the classical syllogism
\begin{quotation}
All men are mortal.\\
Socrates is a man.\\
Therefore, Socrates is mortal.
\end{quotation}
in \Naproche.
\end{exercise}


\subsection{Axiomatic Approach}

The \Naproche{} system
comes with a minimal set of in-built mathematical notions.
Usually one has to explicitly extend the first-order
language through Signature and Definition commands and through
Axioms. Then Lemmas and Theorems can be postulated and proved
with familiar proof structures. In the following this procedure
is explained along a standard proof of the infinitude of
prime numbers:
\begin{itemize}
\item
set up a language and axioms for natural number arithmetic;
\item
define divisibility and prime natural numbers;
\item
introduce some set theory so that one can define
finite sets, sequences and products.
\end{itemize}
Finally, a checked natural language proof of Euclid's theorem
can be carried out in this axiomatic setup.

\section{Importing General Preliminaries}

The \Naproche{} system comes along with a small inbuilt language which
includes the notions of
(mathematical) object, classes, sets, maps and functions:
\begin{quotation}
Elements of classes or sets are objects, maps and functions map objects
to objects. Sets are classes which are objects.
Similarly functions are maps which are objects.
\end{quotation}
Basic properties can be checked in the system. E.g.,
\begin{greybox}
\begin{theorem} Every element of every class is an object. \end{theorem}
\begin{proof} Let $C$ be a class. Let $x$ be an element of $C$.
Then $x$ is an object.
\end{proof}
\end{greybox}

Note that the inbuilt premisses are not available to the external prover
but only to the inbuilt \textit{reasoner} of \Naproche. Since the reasoner is
rather weak, we have to give an explicit proof: instantiate the universal
quantifiers with ``arbitrary but fixed'' instances $C$ and $x$ and prove
the claim in this context. Note that we use the \LaTeX{} proof environment
also as ForTheL proof environment.

The file \path{preliminaries.ftl.tex} proves important properties of the
inbuilt notions, similar to the above example, and it postulates further
axioms about them. More details can be found in the commentary parts of the file.
The file also includes a list \path{vocabulary.ftl} of singular/plural pairs
like the above
\verb+[synonym number/numbers]+, to be used for grammatical correctness.
Moreover, some alternative phrases for certain mathematical phrases are
imported with \path{macros.ftl}.

These preliminaries are imported by:
\begin{forthel}
[readtex preliminaries.ftl.tex]
\end{forthel}

This file also provides the $\subseteq$-relation between classes, so that
we can prove:
\begin{forthel}
\begin{theorem}
Let $C$ be a class. Then $C \subseteq C$.
\end{theorem}
\end{forthel}

\begin{exercise}
Prove that $\subseteq$ is a transitive relation, and that
the empty set is a subclass of every other class.
One can also denote the empty set symbolically by
using the linguistic command:\\
Let $\emptyset$ denote the empty set.
\end{exercise}

\begin{exercise}
The preliminaries file also provides the notion $F(x)$ for elements $x \in \dom(F)$
where $F$ is a function or map.
Formalize the property that a function is injective and that
the composition of two injective functions is injective.
\end{exercise}


\section{On the Translation from Natural to First-Order Language}

Above we have introduced
the functions $+$ and $*$ of addition and multiplication
(of natural numbers)
and the constants $0$ and $1$. Domains are captured by
unary relations. The type ``natural number''
of ordinary mathematical discourse is modeled by the
internal unary relation symbol
\verb+aNaturalNumber+, and the arithmetic functions and
quantifiers are
restricted to the extension of the unary relation symbol.
So the (weak) type system of ordinary mathematical language
is modeled by a system of first-order predicates. These
types do not follow any strict ``type theory'' with specific
mathematical laws but they are powerful enough
to organize the universe of mathematics.

\subsection{First-Order Translation}
The ForTheL code for introducing the type, or rather
notion, of natural numbers, the constants $0$ and $1$
and the operations of $+$ and $*$ has the following
first-order translation
which can be found in the output
window of jEdit or hovering the mouse
over the sentence:
\begin{small}
\begin{verbatim}
1. forall v0 ((HeadTerm :: aNaturalNumber(v0)) implies
     aObject(v0))
2. forall v0 ((HeadTerm :: v0 = 0) implies aNaturalNumber(v0))
3. forall v0 ((HeadTerm :: v0 = 1) implies
     (aNaturalNumber(v0) and not v0 = 0))
4. (aNaturalNumber(m) and aNaturalNumber(n))
5. forall v0 ((HeadTerm :: v0 = m+n) implies aNaturalNumber(v0))
6. (aNaturalNumber(m) and aNaturalNumber(n))
7. forall v0 ((HeadTerm :: v0 = m*n) implies aNaturalNumber(v0))
\end{verbatim}
\end{small}
In these formulas we see the newly introduced first-order symbols:

\verb_aNaturalNumber(v0), 0, 1, +, *_.

The first-order translations follow a certain idiom which
is favourable for the overall processing. Formula 1
exhibits
the new symbol marked by the tag \verb+HeadTerm+. Similarly
formula 2 emphasizes the symbol \verb+0+ which would not have been
the case in the equivalent \verb+aNaturalNumber(0)+.
Note that 5 and 7 both have the premises

\verb+(aNaturalNumber(m) and aNaturalNumber(n))+

\noindent for the two arguments of the operations.

The correctness of mathematical texts includes the principle of
\textit{ontological correctness} which in this case means that
such premises have to
be proved before the operations can reasonably be applied within
a proof. \Naproche{} checks texts sentence by sentence.
The first check is for the indicated ontological correctness
of the sentence. Only if that is successful the logical correctness
will be checked if the sentence is an assertion of a mathematical fact;
Assumptions, or statements in signature or definition commands will
be treated differently.

\begin{exercise}
Extract the first-order translation of one of your formalizations.
You can put the command \verb+[dump on]+ into your ForTheL text
to increase the amount
of information in the output buffer of Isabelle.
\end{exercise}

\subsection{Some ForTheL Commands and Keywords}

Let us now go through the natural language phrases used to
reach this translation. New first-order symbols are spawned by
Signature commands. The new notion comes before the keyword ``is''
after which the new notion is classified as a new type (``is a notion'')
or as a member of of an existing type (``is a natural number'').

The phrase before ``is'' is read as a new language pattern that
the parser learns. A pattern has some word tokens,
like ``natural'',
``number'', or some symbolic tokens, like
``$0$'', ``$1$'', ``$+$'', ``$*$''.
In between those tokens a pattern may have holes for the insertion
of terms, which in the Signature command are indicated by previously
introduced variables, like ``$m$'' or ``$n$''. These were introduced in
the parser command ``Let $m,n$ denote natural numbers.''
Thereafter, $m$ and $n$ are variables which are ``pretyped'' to be
natural numbers. With that,
\begin{signature} $m + n$ is a natural number.
\end{signature}
has the ``double translation''
\begin{verbatim}
(aNaturalNumber(m) and aNaturalNumber(n))
forall v0 ((HeadTerm :: v0 = m+n) implies aNaturalNumber(v0))
\end{verbatim}
where the first (or more) formulas are premises and the last contains the
newly introduced symbol.

We can also qualify the typing on the right-hand side of the
``is'' keyword
by first-order formulae. In our example, we have introduced a pattern for
a first-order formula by the parser command
``Let $x$ is nonzero stand for $x \neq 0$.''
This formula is then applied
as an adjective in the next Signature command
\begin{signature} $1$ is a nonzero natural number.
\end{signature}
Note that some natural language processing is also taking place:
``nonzero'' is introduced within the phrase ``$x$ is nonzero''
in an
adjective position. So in the Signature command, ``nonzero''
can be
used as an adjective which modifies ``natural number''.
The first-order
effect of this is a conjunction
\begin{verbatim}
3. forall v0 ((HeadTerm :: v0 = 1) implies
     (aNaturalNumber(v0) and not v0 = 0))
\end{verbatim}
The equality ``$=$'' and inequality ``$\neq$'' are predefined
phrases with corresponding first-order symbols.

\subsection{``Grammar''}

Note that we have also used``linguistic'' commands:
the command \verb+[synonym number/numbers]+
identifies the tokens ``number'' and ``numbers'', providing the
plural form. The command can be abbreviated to
\verb+[synonym number/-s]+. This is a simple linguistic
``hack'' which allows grammatically correct forms. But it
also allows wrong ones, and it is up to the user to make the right
choices.

\section{Postulating Natural Number Axioms}

We need to introduce axioms for our abstract first-order structure.
Axioms are ForTheL statements written in axiom environments.
For arithmetic we use self-explanatory symbolic formulas.
There are many ways of axiomatizing the natural numbers in order
to be able to prove our final goal: the infinitude of
primes. Here we axiomatize the natural numbers as
a sort of commutative ``half-ring'' with $1$.
We provide ourselves with another variable $l$,
pretyped as a natural number.

\begin{forthel}
Let $m,n$ stand for natural numbers.

\begin{axiom} $m + n = n + m$.
\end{axiom}

Let $l$ stand for a natural number.

\begin{axiom} $(m + n) + l = m + (n + l)$.
\end{axiom}

\begin{axiom}  $m + 0 = m = 0 + m$.
\end{axiom}

\begin{axiom} $m * n = n * m$.
\end{axiom}

\begin{axiom} $(m * n) * l = m * (n * l)$.
\end{axiom}

\begin{axiom} $m * 1 = m = 1 * m$.
\end{axiom}

\begin{axiom} $m * 0 = 0 = 0 * m$.
\end{axiom}

\begin{axiom} $m * (n + l) = (m * n) + (m * l)$ and
                $(n + l) * m = (n * m) + (l * m)$.
\end{axiom}

\begin{axiom} If $l + m = l + n$ or $m + l = n + l$
then $m = n$.
\end{axiom}

\begin{axiom} Assume that $l$ is nonzero.
If $l * m = l * n$ or $m * l = n * l$ then $m = n$.
\end{axiom}

\begin{axiom} If $m + n = 0$ then $m = 0$ and $n = 0$.
\end{axiom}

\end{forthel}

Axioms - like Signatures - are toplevel sections which consist of
$n + 1$ statements. The first $n$ are assumption statements
(``Assume ...'', ``Let ...'')
under which the final statement is postulated. Note that
pretypings of variables also act like assumptions.

\section{The Natural Order - Defining \\Relations and Functions}

Definitions extend the first-order language by defined symbols
as in the following examples concerning the ordering
of the natural numbers. A definition corresponds to a
Signature command in which a symbol is introduced plus
an Axiom containing the defining property.

\begin{forthel}
Let $m,n$ stand for natural numbers.

\begin{definition} $m \leq n$ iff
there exists a natural number $l$ such that
$m + l = n$.
\end{definition}

Let $m < n$ stand for $m \leq n$ and $m \neq n$.

\begin{definition} Assume that $n \leq m$.
$m - n$ is a natural number $l$ such that $n + l  = m$.
\end{definition}

\end{forthel}

The first definition defines the binary relation $\leq$ by an
``iff'' equivalence. This is followed by a purely
syntactic definition of $<$. $m < n$ is simply an abbreviation
for another formula. Such abbreviations and aliases are already expanded
during parsing, possibly recursively. The third definition
defines the binary difference function $-$.

\begin{exercise}
Introduce aliases for numbers and arithmetic operations
so that one can prove:
``The sum of one and three is equal to five''.
\end{exercise}

\begin{exercise}
Introduce the square $m^2$ of a natural number and
prove the binomial formula for $(m+n)^2$.
\end{exercise}

\subsection{Axiomatic Content in Definitions}

Definitions of functions and constants usually contain
implicit postulates corresponding to the existence
and uniquess-properties of function values and constants. In
the above definition of the difference ($-$), the condition for 
$l$ should be
satisfiable by a unique natural number. This is however \textit{not}
checked by \Naproche, so that the well-definedness of the
function is the user's responsibility. If the function
definition were non-unique we could have a contradictory
system of assumptions. Consider, e.g., the wrong definition

\begin{definition} Assume that $n \leq m$.
$m - n$ is a natural number $l$ such that $n = m$.
\end{definition}

The first-order translation would be

\begin{verbatim}
(aNaturalNumber(m) and aNaturalNumber(n))
n\leq m
forall v0 ((HeadTerm :: v0 = m-n)
   iff (aNaturalNumber(v0) and n = m))
\end{verbatim}

Every number fits the defining equivalence provided that $m = n$.
But then $0 = 0 - 0 = 1$, contradiction.

With relation definitions, such problems do not arise.

\section{Lemmas and Theorems}

After setting up the axiomatics we proceed to claim and prove
propositions. Claims together with the accumulated facts are
given to the background ATP (= eprover). Many basic
propositions can be proved by the ATP without further intervention.
The following three lemmas show that $\leq$ is a partial order:

\begin{forthel}
Let $l,m,n$ stand for natural numbers.

\begin{lemma} $m \leq m$.
\end{lemma}

\begin{axiom} If $m \leq n \leq m$
then $m = n$.
\end{axiom}

\begin{lemma} If $m \leq n \leq l$
then  $m \leq l$.
\end{lemma}
\end{forthel}

\subsection{Eprover in the Background}
These lemmas are checked correct by \Naproche{} without explicit proofs.
We can look at the tasks given to the ATP by putting
a [dump on] command in the beginning of the ForTheL parts of the
document and looking for the dump of the provertask in the
output window. The task is written in the first-order
logic language TPTP which is a standard input language for ATPs.
Observe that all previous Signature, Axiom and Definition
commands can be found as premises of the conjecture.

The following is part of the trace of proving $m \leq m$. Actually
the ontological correctness of the formula is checked first.
In case of the $\leq$-relation, the assumptions or presuppositions
were given by the pretypings of both arguments. So for the
formula $m \leq m$ we have to show that $m$ is a natural number.
This prover task is expressed by the following
TPTP text:
\begin{scriptsize}
\begin{verbatim}
[Translation] (line 409 of ...
aNaturalNumber(m)
[Translation] (line 409 of ...
m\leq m
[Reasoner] (line 409 of ...
goal:  m \leq m .
[Main] (line 409 of ...
fof(m_,hypothesis,$true).
fof(m_,hypothesis,aNaturalNumber(sz0)).
fof(m_,hypothesis,(aNaturalNumber(sz1) & ( ~ (sz1 = sz0)))).
fof(m_,hypothesis,( ! [W0] : ( ! [W1] : ((aNaturalNumber(W0) & aNaturalNumber(W1))
  => aNaturalNumber(sdtpldt(W0,W1)))))).
fof(m_,hypothesis,( ! [W0] : ( ! [W1] : ((aNaturalNumber(W0) & aNaturalNumber(W1))
  => aNaturalNumber(sdtasdt(W0,W1)))))).
fof(m_,hypothesis,( ! [W0] : ( ! [W1] : ((aNaturalNumber(W0) & aNaturalNumber(W1))
 => (sdtpldt(W0,W1) = sdtpldt(W1,W0)))))).
fof(m_,hypothesis,( ! [W0] : ( ! [W1] : ( ! [W2] : (((aNaturalNumber(W0) &
  aNaturalNumber(W1)) & aNaturalNumber(W2))
  => (sdtpldt(sdtpldt(W1,W2),W0) = sdtpldt(W1,sdtpldt(W2,W0)))))))).
fof(m_,hypothesis,( ! [W0] : (aNaturalNumber(W0)
  => ((sdtpldt(W0,sz0) = W0) & (W0 = sdtpldt(sz0,W0)))))).
fof(m_,hypothesis,( ! [W0] : ( ! [W1] : ((aNaturalNumber(W0) & aNaturalNumber(W1))
  => (sdtasdt(W0,W1) = sdtasdt(W1,W0)))))).
fof(m_,hypothesis,( ! [W0] : ( ! [W1] : ( ! [W2] : (((aNaturalNumber(W0) &
  aNaturalNumber(W1)) & aNaturalNumber(W2))
  => (sdtasdt(sdtasdt(W1,W2),W0) = sdtasdt(W1,sdtasdt(W2,W0)))))))).
fof(m_,hypothesis,( ! [W0] : (aNaturalNumber(W0)
  => ((sdtasdt(W0,sz1) = W0) & (W0 = sdtasdt(sz1,W0)))))).
fof(m_,hypothesis,( ! [W0] : (aNaturalNumber(W0)
  => ((sdtasdt(W0,sz0) = sz0) & (sz0 = sdtasdt(sz0,W0)))))).
fof(m_,hypothesis,( ! [W0] : ( ! [W1] : ( ! [W2] : (((aNaturalNumber(W0)
  & aNaturalNumber(W1)) & aNaturalNumber(W2))
  => ((sdtasdt(W1,sdtpldt(W2,W0)) = sdtpldt(sdtasdt(W1,W2),sdtasdt(W1,W0)))
  & (sdtasdt(sdtpldt(W2,W0),W1) = sdtpldt(sdtasdt(W2,W1),sdtasdt(W0,W1))))))))).
fof(m_,hypothesis,( ! [W0] : ( ! [W1] : ( ! [W2] : (((aNaturalNumber(W0)
  & aNaturalNumber(W1)) & aNaturalNumber(W2))
  => (((sdtpldt(W0,W1) = sdtpldt(W0,W2))
  | (sdtpldt(W1,W0) = sdtpldt(W2,W0))) => (W1 = W2))))))).
fof(m_,hypothesis,( ! [W0] : (aNaturalNumber(W0) => (( ~ (W0 = sz0))
  => ( ! [W1] : ( ! [W2] : ((aNaturalNumber(W1) & aNaturalNumber(W2))
  => (((sdtasdt(W0,W1) = sdtasdt(W0,W2))
  | (sdtasdt(W1,W0) = sdtasdt(W2,W0))) => (W1 = W2))))))))).
fof(m_,hypothesis,( ! [W0] : ( ! [W1] : ((aNaturalNumber(W0) & aNaturalNumber(W1))
  => ((sdtpldt(W0,W1) = sz0) => ((W0 = sz0) & (W1 = sz0))))))).
fof(m_,hypothesis,( ! [W0] : ( ! [W1] : ((aNaturalNumber(W0) & aNaturalNumber(W1))
  => (sdtbszlzezqdt(W0,W1)
  <=> ( ? [W2] : (aNaturalNumber(W2) & (sdtpldt(W0,W2) = W1)))))))).
fof(m_,hypothesis,( ! [W0] : ( ! [W1] : ((aNaturalNumber(W0) & aNaturalNumber(W1))
  => (sdtbszlzezqdt(W1,W0) => ((aNaturalNumber(sdtmndt(W0,W1))
  & (sdtpldt(W1,sdtmndt(W0,W1)) = W0)) & ( ! [W2] : ((aNaturalNumber(W2)
  & (sdtpldt(W1,W2) = W0)) => (W2 = sdtmndt(W0,W1)))))))))).
fof(m__,hypothesis,aNaturalNumber(xm)).
\end{verbatim}
\end{scriptsize}

\subsection{Testing for Contradictions}

It is quite common to accidentally introduce trivial inconsistencies
in formalizations. Not just by function definitions, but also
because some marginal cases outside the main argument have not
been treated right. E.g., although the number $0$ is quite
uninteresting for the study of prime numbers, we still have to
deal with $0$-cases or explicitly request that terms are
nonzero. If a text with non-trivial mathematical content checks
unexpectedly fast then one should become suspicious, because 
eprover may exploit a contradiction to prove every proposition via
\textit{ex falso quodlibet}.

To find inconsistencies it is helpful to try to prove
\begin{lemma} Contradiction. \end{lemma}
\noindent in various places of a text. If the lemma is validated by
\Naproche{} then one has to investigate further. Contradiction-
lemmas can be quickly de-activated by commenting out
with \verb+%+.. They can also be used to force rechecking of 
the text: uncomment
the lemma and then comment it again; this will lead to rechecking 
from the position of the lemma onwards.


\section{Linear and Discrete Orders}

We need more axiomatic assumptions for the ordering of the natural numbers. The axioms
so far do not guarantee that the ordering is linear. 
So we continue:

\begin{forthel}
Let $l,m,n$ stand for natural numbers.

\begin{axiom} $m \leq n$ or $n < m$.
\end{axiom}

\begin{lemma} Assume that $l < n$.
  Then $m + l < m + n$ and $l + m < n + m$.
\end{lemma}

\begin{lemma} Assume that $m$ is nonzero and $l < n$.
  Then $m * l < m * n$ and $l * m < n * m$.
\end{lemma}

\end{forthel}

\section{Induction}

\Naproche{} has inherited an elegant treatment of induction
from the SAD system. \Naproche{} has a special binary relation
symbol $\prec$ for a universal inductive relation: if
the property $P$ is \textit{inductive} along $\prec$, i.e., if at any
point $m$ property $P$ is inherited at $m$ provided all
$\prec$-predecessors of $m$ satisfy $P$, then $P$ holds everywhere.
More formally the general induction principle for the relation 
$\prec$ reads:
\begin{quotation} 
To prove $\forall k \phi(k)$, it suffices to prove
inductivity along $\prec$:
$$\forall v_0 (\forall v_1 (v_1 \prec v_0 \rightarrow \phi(v_1))
\rightarrow \phi(v_0).$$
\end{quotation}

This modification of proof tasks
is immediately carried out by the parser when it comes across the
keyword ``proof by induction''.

Induction along the relation $<$ on natural numbers is introduced
axiomatically by embedding the $<$-relation into $\prec$:

\begin{forthel}
Let $m,n$ stand for natural numbers.

\begin{axiom} If $n < m$ then $n \prec m$.
\end{axiom}
\end{forthel}

Let us consider a proof by induction that $\leq$ is
a ``discrete'' order with nothing strictly between
$0$ and $1$.

\begin{forthel}
\begin{lemma} For every $n$ $n = 0$ or $1 \leq n$.
\end{lemma}
\begin{proof}[by induction on $n$]
\end{proof}
\end{forthel}

The automatic translation of the statement of the lemma is
\begin{verbatim}
forall v0 (aNaturalNumber(v0) implies
((InductionHypothesis ::
forall v1 (aNaturalNumber(v1) implies (iLess(v1,v0) implies
(v1 = 0 or 1\leq v1))))
implies (v0 = 0 or 1\leq v0)))
\end{verbatim}

The formula expresses that the property ``$n = 0$ or $1 \leq n$''
is \textit{inductive}. Proving the universal formula
``For every $n$ $n = 0$ or $1 \leq n$'' has been reduced to
proving inductivity. In this case the inductivity task can be 
carried out automatically by
eprover.

Remark: It seems difficult to prove this inductivity by a short intuitive
proof. Apparently the above axiomatization is not standard and should
be replaced by an approach where the natural numbers are generated from
$0$ by a $+1$-operation, expressed by using $n \prec n+1$ as
induction axiom.

With the previous lemma we can show:

\begin{forthel}
\begin{lemma} If $m \neq 0$ then $n \leq n * m$.
\end{lemma}
\end{forthel}

\section{Division}

Prime (natural) numbers are defined via divisibility.

\begin{forthel}
Let $l,m,n$ stand for natural numbers.

\begin{definition}
  $n$ divides $m$ iff for some $l$ $m = n * l$.
\end{definition}

Let $x | y$ denote $x$ divides $y$.
Let a divisor of $x$ denote a natural number
that divides $x$.

\begin{lemma} Assume $l | m | n$. Then $l | n$.
\end{lemma}

\end{forthel}

The definition is similar to the definition of $\leq$.
Note, however, the possible syntactic variations:
``there exists a natural number $l$ such that
$m = n * l$''; it is also possible
to put the quantifier after the property:
``$n$ divides $m$ iff $m = n * l$ for some $l$''.

Natural language has many mechanisms for putting
information into sentences in a compact, un-formalistic way.
Un-formalistic means, e.g., that natural language
does not normally use brackets (...) in speech.
``The syntax and semantics of the ForTheL language'' by Andrei
Paskevich is still a good guide to most ForTheL constructs, although
the language has been extended since then.

\section{An Interactive Proof}

The final contradiction in Euclid's proof will need:

\begin{lemma} Let $l | m$ and $l | m + n$. Then $l | n$.
\end{lemma}

\Naproche{} does not find
a proof on its own: depending on some default timeouts the
proof search is abandoned, and the goal $l | n$ fails.
In Isabelle-Naproche this is signaled in the output
window, and the failed goal is underlined in red.

So the user has to ``interactively'' supply a proof, which in a first
approximation is a list
of statements leading up to the claim, and which
\Naproche{}'s ATP is able to prove successively.
Proof statements can also introduce assumptions
and new variables to the argument, and they can
structure the proof.

\begin{forthel}
Let $l,m,n$ stand for natural numbers.

\begin{lemma} Let $l | m$ and $l | m + n$. Then $l | n$.
\end{lemma}

\begin{proof}
Assume that $l$ is nonzero.
Take a natural number $p$ such that $m = l * p$.
Take a natural number $q$ such that $m + n = l * q$.

Let us show that
$p \leq q$.
Proof by contradiction.
Assume the contrary. Then $q < p$.
$m+n = l * q < l * p = m$.
Contradiction. qed.

Take $r = q - p$.
We have $(l * p) + (l * r) = l * q = m + n = (l * p) + n$.
Hence $n = l * r$.
\end{proof}

\end{forthel}

When \Naproche{} encounters a statement immediately followed by an
explicit proof
then \Naproche{} defers proving the statement and first goes through
the proof. Since proofs may contain subproofs, this process may
take place recursively.

Proofs of a ``toplevel'' Lemma or Theorem use the

\verb+\begin{proof}...\end{proof}+

\noindent environment well-known from \LaTeX.
In our proof there is also a ``lowlevel'' proof of $p \leq q$
indicated by ``Let us show that''. We discuss some aspects of the
proof:
\begin{itemize}
\item Most sentences in a proof are statements, or statements
extended by constructs that organize the flow of the argument.
\item ``Assume that $l$ is nonzero.'' is an assumption that introduces
the premise ``$l$ is nonzero'' to the argument. Instead of ``Assume that''
one could also use variants like
[let us $\mid$ we can] (assume $\mid$ presume $\mid$ suppose) [that].
\item ``Take $p$ such that $m = l * p$.'' introduces a new variable
$p$ with a specific property to the argument. To verify this construct
the prover has to show the existence of some object satisfying the
property. Again there are variants:
[let us $\mid$ we can] (choose $\mid$ take $\mid$ consider).
\item ``Let us show that $p \leq q$.'' claims that the statement
$p \leq q$ holds and announces a subsequent proof.
Alternatives: [let us $\mid$ we can] 
(prove $\mid$ show $\mid$ demonstrate) (that).
\item ``Proof by contradiction'' denotes the start of an indirect
proof. It is recommended to explicitly mark indirect proofs.
Note that in the example there is a ``lowlevel'' proof that
uses a simple

\verb+Proof [by ...](.) ... (qed. | end.)+

\noindent environment instead of the \LaTeX proof environment.
\item Other proof methods are ``by cases'' and ``by induction''.
\item ``Assume the contrary.'': The contrary is the negation
of the current thesis which in this case is the statement claimed
just before. ``thesis'' denotes the current thesis, ``contradiction''
stands for ``false''.
\item ``Then $q < p$.'': Words like ``then'', ``hence'', ``thus'',
``therefore'', ``consequently'' are filler words which are redundant
for \Naproche{} but may help human readers to understand the text.
\item ``$m+n = l * q < l * p = m$'': binary relations like ``$=$''
or ``$<$'' can be chained. The statement means the conjunction
of the single relations. These will be checked from
left to right.
\item ``Contradiction. qed.'': The indirect proof has reached the
desired contradiction, and the (local) proof environment is closed by
``qed.''.
\end{itemize}

\Naproche{} is able to prove the next lemma without an
explicit proof in the text.

\begin{forthel}

\begin{lemma} Let $m | n \neq 0$. Then $m \leq n$.
\end{lemma}
\end{forthel}

\section{Primes}

Prime numbers are defined as usual. Indeed we define
the adjective ``prime'' which will enable us
to write ``prime natural number'' or ``prime divisor''.

\begin{forthel}
Let $m,n$ stand for natural numbers.
Let $x$ is nontrivial stand for $x \neq 0$ and $x \neq 1$.

\begin{definition}
$n$ is prime iff $n$ is nontrivial and
    for every divisor $m$ of $n$ $m = 1$ or $m = n$.
\end{definition}

\end{forthel}

The following lemma obviously holds by induction: either
$k$ is prime itself, or $k$ has a divisor strictly
between $1$ and $k$; by induction that divisor has a prime
divisor which is also a prime divisor of $k$.

\begin{forthel}
%[prove off]
\begin{lemma} Every nontrivial $m$ has a prime divisor.
\end{lemma}
\begin{proof}[by induction on $m$] \end{proof}
\end{forthel}


``Proof by induction'' transforms the thesis into the inductivity
theses:
\begin{footnotesize}
\begin{verbatim}
thesis: forall v0 ((aNaturalNumber(v0) and (not v0 = 0 and not v0 = 1))
implies ((InductionHypothesis :: forall v1 ((aNaturalNumber(v1) and
(not v1 = 0 and not v1 = 1)) implies (iLess(v1,v0)
implies exists v2 ((aNaturalNumber(v2) and doDivides(v2,v1))
and isPrime(v2))))) implies exists v1
((aNaturalNumber(v1) and doDivides(v1,v0)) and isPrime(v1))))
\end{verbatim}
\end{footnotesize}
which can be discharged automatically by eprover.

\section{Sets and Classes}

Some basic properties of sets and classes are built into \Naproche{}
or are explicitly postulated in the \path{preliminaries.ftl.tex} file.
We shall use these to express the infinitude of the collection of
prime natural numbers.

Arbitrary collections of objects can be defined naturally in ForTheL:

\begin{forthel}
\begin{definition} $\mathbb{N}$ is the
collection of all natural numbers.
\end{definition}
\end{forthel}

This translates as
\begin{verbatim}
forall v0 ((HeadTerm :: v0 = \mathbb{N}) iff
(aClass(v0) and forall v1
(aElementOf(v1,v0) iff (aNaturalNumber(v1) and aObject(v1)))))
\end{verbatim}

The \verb+aObject(v1)+ is part of the collection construct and ensures
that only objects will be elements of the collection. Such collections
are automatically registered as {\em classes} (\verb+aClass(v0)+).
There are a few grammatical extensions and variants of this construct:
in particular one could form the collection of all
[terms] such that [condition], where [condition] is an arbitrary ForTheL
statement.

Classes can also be defined by abstraction terms, i.e., we could
have defined alternatively:

\begin{definition}
$\mathbb{N} = \{n | n$ is a natural number$\}$.
\end{definition}

To work freely with the class of natural numbers one wants it be a
{\em mathematical object} itself, i.e., a {\em set}. The axioms so far do not
imply this, as one can check by trying to prove
``$\mathbb{N}$ is a set.''
So we postulate this as an axiom which is (equivalent to) the
well-known axiom of infinity in set theory:

\begin{forthel}
\begin{axiom} $\mathbb{N}$ is a set.
\end{axiom}
\end{forthel}

We define the set of prime numbers so that we can prove its infinity.

\begin{forthel}
\begin{definition}
$\mathbb{P} = \{n | n$ is a prime natural number$\}$.
\end{definition}
\end{forthel}
The axiom of separation from \path{preliminaries.ftl.tex} yields:
\begin{forthel}
\begin{lemma}
$\mathbb{P}$ is a set.
\end{lemma}
\end{forthel}
\noindent since $\mathbb{P}$ is a subclass of the {\em set} $\mathbb{N}$.

\begin{exercise}
Form the collection of all even numbers or of all square numbers and
prove that these are sets.
\end{exercise}

\begin{exercise}
Prove that the class $\{ x | x$ is a set and $x \notin x \}$ is \textit{not}
a set.
\end{exercise}

\section{Finite Sequences and Products}

We shall use natural numbers and bounded intervals
of natural numbers to ``count'' the elements of sets.

\begin{forthel}
Let $m,n$ stand for natural numbers.

\begin{definition} $\Seq{m}{n} = \{ i \in \mathbb{N} |
m \leq i \leq n \}$.
\end{definition}
\end{forthel}

This translates to
\begin{verbatim}
forall v0 ((HeadTerm :: v0 = \Seq{m}{n}) iff ((aClass(v0) and
(aSet(\mathbb{N}) implies aSet(v0))) and
forall v1 (aElementOf(v1,v0) iff ((aElementOf(v1,\mathbb{N}) 
and (m\leq v1 and v1\leq n)) and aObject(v1)))))
\end{verbatim}

Note that the second line of the translation contains an instance
of the axiom of separation: if the class that bounds 
the variable $i$ is a set then
the resulting class is a set. Thus:

\begin{forthel}
\begin{lemma} $\Seq{m}{n}$ is a set.
\end{lemma}
\end{forthel}

The notation $\Seq{m}{n}$ indicates the notational and 
typographical potential
for \Naproche{} texts if one uses appropriate \LaTeX macros.

$\Seq{m}{n}$ is printed using the macro

\verb+\newcommand{\Seq}[2]{\{#1,\dots,#2\}}+

The macro is used in the source code of the above definition:

\begin{verbatim}
\begin{definition} $\Seq{m}{n} = \{ i \in \mathbb{N} |
m \leq i \leq n \}$.
\end{definition}
\end{verbatim}

\Naproche{} reads this as the introduction of a binary function
pattern \verb+\Seq{m}{n}+ as one can see from the first-order translation.

In this way, intuitive and customary $\dots$-notations which may seem
vague can be given an unambiguous meaning and can be used
in \Naproche{} formalizations.

These techniques are used in further definitions and notations for finite sequences.

\begin{forthel}

\begin{definition} A sequence of length $n$ is a
function $F$ such that $\dom(F) = \Seq{1}{n}$.
\end{definition}

\end{forthel}

The members $F(i)$ of a sequence $F$ are often
written in an indexed notation $F_{i}$.
This is just an alias at the parsing level:

\begin{forthel}
Let $F_{i}$ stand for $F(i)$.

\begin{definition} Let $F$ be a sequence of length $n$.
$\Finset{F}{1}{n} = \{ F_{i} | i \in \dom(F)\}$.
\end{definition}
\end{forthel}

This definition uses the \LaTeX{}-macro

\verb+\newcommand{\Finset}[3]{\{#1_{#2},\dots,#1_{#3}\}}+

Dot notation is also used for iterations of all sorts.
For Euclid's theorem we shall want to consider products
of finitely many prime numbers. So we postulate
axiomatically:

\begin{forthel}

\begin{signature} Let $F$ be a sequence of length $n$
such that $\Finset{F}{1}{n} \subseteq \mathbb{N}$.
$\Prod{F}{1}{n}$ is a natural number.
\end{signature}

\begin{axiom}[Factorproperty] Let $F$ be a sequence of length $n$
such that $F(i)$ is a nonzero natural number for every $i \in \dom(F)$.
Then $\Prod{F}{1}{n}$ is nonzero and
$F(i)$ divides $\Prod{F}{1}{n}$ for every $i \in \dom(F)$.
\end{axiom}


\end{forthel}

Note that we can name toplevel sections by single words like ``Factorproperty''
or numbers. These can be referenced later in the form ``(by Factorproperty)''.

\section{Finite and Infinite Sets}

Finite sequences readily allow a formalization
of finiteness for arbitrary sets and classes.

\begin{forthel}

Let $S$ denote a class.

\begin{definition} $S$ is finite iff
$S = \Finset{F}{1}{n}$ for some natural number $n$ and some function $F$ that is
a sequence of length $n$.
\end{definition}

\begin{definition} $S$ is infinite iff $S$ is not finite.
\end{definition}

\end{forthel}

\section{Euclid's Theorem}

Now everything is in place for the proof that there
are infinitely many prime numbers.
\begin{forthel}

\begin{signature} $\Primes$ is the class of prime natural numbers.
\end{signature}

% \begin{lemma} Contradiction. \end{lemma}


\begin{theorem}[Euclid]
$\Primes$ is infinite.
\end{theorem}
\begin{proof}
Assume that $r$ is a natural number and
$p$ is a sequence of length $r$ and
$\Finset{p}{1}{r}$ is a subclass of $\Primes$.

(1) $p_{i}$ is a nonzero natural number for every $i \in \dom(p)$.

Consider $n=\Prod{p}{1}{r}+1$.
$\Prod{p}{1}{r}$ is nonzero.
Hence $n$ is nontrivial. Take a prime divisor $q$ of $n$.

Let us show that $q \neq p_{i}$ for all natural numbers $i$ such that
$1 \leq i \leq r$.

Proof by contradiction.
Assume the contrary.
Take a natural number $i$ such that $1 \leq i \leq r$ and $q=p_{i}$.
$q$ is a divisor of $n$ and $q$ is a divisor of $\Prod{p}{1}{r}$
(by Factorproperty,1).
Thus $q$ divides $1$. Contradiction. qed.

Hence $\Finset{p}{1}{r}$ is not the class of prime natural numbers.
\end{proof}
\end{forthel}

\end{document}
